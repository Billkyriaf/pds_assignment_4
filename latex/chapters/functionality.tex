\subsubsection*{Pthreads functionality}

The pthreads framework provide maximum functionality. Everything is possible.
This complexity though comes with a price. The pthreads framework is complex
and requires a lot of work on the part of the programmer. The programmer on the
other hand has maximum control over the program. If designed correctly the program
can be very efficient and achieve great performance.

The pthreads framework is a well established library with plenty of documentation,
tutorials and material available online. This only increases it's popularity and
makes it a very good choice for a parallel programming framework.

\subsubsection*{openCilk functionality}

The openCilk framework is a relatively new framework. It is a set of compiler extensions
that allow the programmer to write parallel programs in a sequential manner. The complexity
of designing the parallel program is hidden from the programmer behind the compiler. The side
effect of this is that the programmer has less control over the program.

For the moment the openCilk framework has not achieved critical mass. It is slowly gaining
popularity but the available documentation and tutorials are not very extensive. As a result
it is can be tricky to get a good understanding of the framework.

Nevertheless, the ease of use of the openCilk framework makes it a very good choice for fast
and easy parallelization of sequential algorithms.
