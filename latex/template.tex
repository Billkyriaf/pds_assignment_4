% summary-pentagram.tex
% 
\documentclass[letterpaper,12pt]{article}

\usepackage[top=1in,bottom=1in, left=1in, right=1in]{geometry}
\usepackage{fancyhdr}
\usepackage{amsmath}
\usepackage{amsfonts,amssymb,amsthm,cite,float,graphicx}
\usepackage[usenames]{color}

\pagestyle{fancy}

\setlength{\headheight}{15.2pt}

\lhead{Project:\phantom{project name}}
\chead{Summary}
\rhead{Author: \phantom{author-name} }

\begin{document}
\section*{Five key components of a project summary}

\vspace{1in}
\begin{enumerate}
\item Problem description and significance in clear statements. 
\item[]
  
\item (Prior) state of the art (SOTA): existing knowledge,
  methodologies, limitation or gaps; formally cite credible and
  updated sources with a bibliography.
\item[]
  
\item The objective and method of this particular project: a novel
  approach, or a novel interpretation, understanding or use of
  existing approaches.
\item[]
  
\item Results. On analytical results, clarify conditions and
  conclusions. On experimental results, specify metrics and design
  plots, tables, maps or images for overview. On program products,
  deliver clean, well-documented, working codes.
\item[]
  
\item Conclusion \& discussion: a summary of key findings, including
  advances made, remaining limitations, or lessons learned (in
  learned lessons, we turn negative results to positive experience and
  knowledge).
  %
\item[ ] Acknowledgement. Formerly acknowledge others for critical
  guidance, suggestions, comments and help, or funding, in brief
  statements.
  
\end{enumerate}

\vspace{3em}

The key elements also apply to a summary slide presentation.  Check
the listed key items before submitting a project summary.  In addition
to the problem and solution, a project summary also reflects the
author's scholarly knowledge, aptitude, skills, experience and
integrity.

\end{document}

%%% Xiaobai Sun
%%% Duke CS